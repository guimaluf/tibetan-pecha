\font\tibetan=tib\font\tibsp=tibsp \documentstyle{article}
\begin{document}
\begin{center}
{\large\bf Using the Tibetan Font with \LaTeX}
\end{center}

\noindent
\LaTeX \ commands are available outside the Tibetan mode only.
The two characters \%\%\ \underline{must} precede
text that is to be processed by the Tibetan Parser.  They effect a font
change into Tibetan, begin spacing and punctuation according to Tibetan
rules, and allow control over output using the special characters
described here.  At the end of Tibetan text, these two characters
\underline{must} appear again.  \LaTeX \ expects white space to
precede and follow these font change characters.

\medskip
\noindent
The following characters are embedded in the text to access
the Tibetan Parser and font:

\begin{description}
\item[/] \hspace{\parindent}  -- the Tibetan {\em shad}: \bgroup\tibetan \def\u#1{\vtop{\baselineskip0pt\hbox{#1}\hbox{\tibsp\char123}}}\hyphenpenalty=10000\parindent=0pt \newbox\fillerbox\setbox\fillerbox\hbox{\vrule height0.4cm depth0.4cm width0pt}\def\filler{\copy\fillerbox}\filler\tibsp\char115\hskip1cm plus1cm minus.5cm\tibetan
\egroup  As
many of these may be used as desired.  The last one followed by white
space produces a long space.
\item[!] \hspace{\parindent} -- the Tibetan {\em tsheg shad}: \bgroup\tibetan \def\u#1{\vtop{\baselineskip0pt\hbox{#1}\hbox{\tibsp\char123}}}\hyphenpenalty=10000\parindent=0pt \newbox\fillerbox\setbox\fillerbox\hbox{\vrule height0.4cm depth0.4cm width0pt}\def\filler{\copy\fillerbox}\filler\tibsp\char121\hskip1cm plus1cm minus.5cm\tibetan
\egroup 
Only one of these may be used at the end of a clause.  It effects the
same long space as the {\em shad}.
\item[,] \hspace{\parindent} -- inserts a {\em tsheg} $\langle$ \bgroup\tibetan \def\u#1{\vtop{\baselineskip0pt\hbox{#1}\hbox{\tibsp\char123}}}\hyphenpenalty=10000\parindent=0pt \newbox\fillerbox\setbox\fillerbox\hbox{\vrule height0.4cm depth0.4cm width0pt}\def\filler{\copy\fillerbox}\tibsp\char114\tibetan%
\filler\tenrm\ \tibetan
\egroup 
$\rangle$ into the text.  The $\langle$ \bgroup\tibetan \def\u#1{\vtop{\baselineskip0pt\hbox{#1}\hbox{\tibsp\char123}}}\hyphenpenalty=10000\parindent=0pt \newbox\fillerbox\setbox\fillerbox\hbox{\vrule height0.4cm depth0.4cm width0pt}\def\filler{\copy\fillerbox}\tibsp\char114\tibetan%
\filler\tenrm\ \tibetan
\egroup  $\rangle$  is
automatically inserted at the end of words of Tibetan text.  The `,' is
useful for inserting a $\langle$ \bgroup\tibetan \def\u#1{\vtop{\baselineskip0pt\hbox{#1}\hbox{\tibsp\char123}}}\hyphenpenalty=10000\parindent=0pt \newbox\fillerbox\setbox\fillerbox\hbox{\vrule height0.4cm depth0.4cm width0pt}\def\filler{\copy\fillerbox}\tibsp\char114\tibetan%
\filler\tenrm\ \tibetan
\egroup  $\rangle$ following \bgroup\tibetan \def\u#1{\vtop{\baselineskip0pt\hbox{#1}\hbox{\tibsp\char123}}}\hyphenpenalty=10000\parindent=0pt \newbox\fillerbox\setbox\fillerbox\hbox{\vrule height0.4cm depth0.4cm width0pt}\def\filler{\copy\fillerbox}\char3\filler\tenrm\ \tibetan
\egroup 
at the end of a clause: \ \bgroup\tibetan \def\u#1{\vtop{\baselineskip0pt\hbox{#1}\hbox{\tibsp\char123}}}\hyphenpenalty=10000\parindent=0pt \newbox\fillerbox\setbox\fillerbox\hbox{\vrule height0.4cm depth0.4cm width0pt}\def\filler{\copy\fillerbox}\char10\char3\tibsp\char114\tibetan%
\filler\tibsp\char115\hskip1cm plus1cm minus.5cm\tibetan
\egroup 
\item[\#] \hspace{\parindent} --  defeats the automatic word marker
$\langle$ \bgroup\tibetan \def\u#1{\vtop{\baselineskip0pt\hbox{#1}\hbox{\tibsp\char123}}}\hyphenpenalty=10000\parindent=0pt \newbox\fillerbox\setbox\fillerbox\hbox{\vrule height0.4cm depth0.4cm width0pt}\def\filler{\copy\fillerbox}\tibsp\char114\tibetan%
\filler\tenrm\ \tibetan
\egroup  $\rangle$  at the end of a word.  It is useful for
Tibetan numbers in text $\langle$ \bgroup\tibetan \def\u#1{\vtop{\baselineskip0pt\hbox{#1}\hbox{\tibsp\char123}}}\hyphenpenalty=10000\parindent=0pt \newbox\fillerbox\setbox\fillerbox\hbox{\vrule height0.4cm depth0.4cm width0pt}\def\filler{\copy\fillerbox}\tibsp\char1\tibetan%
\tibsp\char2\tibetan%
\tibsp\char3\tibetan%
\tibsp\char4\tibetan%
\filler\tenrm\ \tibetan
\egroup  $\rangle$ and producing
isolated characters.
\item[$\mid$] \hspace{\parindent}  -- produces a long space without the
{\em shad}.  It is useful where \bgroup\tibetan \def\u#1{\vtop{\baselineskip0pt\hbox{#1}\hbox{\tibsp\char123}}}\hyphenpenalty=10000\parindent=0pt \newbox\fillerbox\setbox\fillerbox\hbox{\vrule height0.4cm depth0.4cm width0pt}\def\filler{\copy\fillerbox}\char2\filler\tenrm\ \tibetan
\egroup  ends a clause: \ \bgroup\tibetan \def\u#1{\vtop{\baselineskip0pt\hbox{#1}\hbox{\tibsp\char123}}}\hyphenpenalty=10000\parindent=0pt \newbox\fillerbox\setbox\fillerbox\hbox{\vrule height0.4cm depth0.4cm width0pt}\def\filler{\copy\fillerbox}\char22\u{\char10}\char2\filler\hskip1cm plus1cm minus.5cm\tibetan
\egroup 
{\em Shad} characters may be inserted at the beginning of
the next word using `/': \  \bgroup\tibetan \def\u#1{\vtop{\baselineskip0pt\hbox{#1}\hbox{\tibsp\char123}}}\hyphenpenalty=10000\parindent=0pt \newbox\fillerbox\setbox\fillerbox\hbox{\vrule height0.4cm depth0.4cm width0pt}\def\filler{\copy\fillerbox}\tibsp\char115\tibetan\char108\char25\filler\tibsp\char114\tenrm\ \tibetan
\tibsp\accent125\tibetan\char12\filler\tibsp\char114\tenrm\ \tibetan
\egroup 
\item[$\backslash$ swasti] -- produces \bgroup\tibetan \def\u#1{\vtop{\baselineskip0pt\hbox{#1}\hbox{\tibsp\char123}}}\hyphenpenalty=10000\parindent=0pt \newbox\fillerbox\setbox\fillerbox\hbox{\vrule height0.4cm depth0.4cm width0pt}\def\filler{\copy\fillerbox}\tibsp\char116\tibetan%
\filler\tenrm\ \tibetan
\egroup 
\end{description}

\noindent
The normal way of marking paragraphs in \LaTeX \ is to double space.  The
effect of this in the Tibetan mode is to terminate a line and begin a
new line.  This produces a ragged right and is useful for making lists
in Tibetan.  \end{document}
