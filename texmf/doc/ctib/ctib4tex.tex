\documentclass[a4paper,11pt]{article}
%\usepackage[latin1]{mls}
\usepackage{ctib}
%\usepackage

\newcommand\exa{\nopagebreak \begin{flushleft}\smallskip \nopagebreak
                \begin{minipage}[t]{6cm}\sloppy}
\newcommand\exb{\end{minipage}\kern 1cm\begin{minipage}[t]{8cm}\sloppy }
\newcommand\exc{\end{minipage}\kern -3cm \smallskip\end{flushleft}}

\title{\TibTeX\ \CtibVersionRelease:\\Tibetan for \TeX\ and \LaTeXe}
\author{Oliver Corff\thanks{This work was partially funded by
	DFG (Deutsche Forschungsgemeinschaft) in context with
	the Mongolian Lexicography Project.}\\
	\texttt{corff@zedat.fu-berlin.de}}

\date{July 1st, 2001}

\begin{document}
\maketitle
\begin{abstract}
	\TibTeX\ is a package offering Tibetan support for \TeX\ and
	\LaTeXe.  This package is based on earlier works by Schwartz,
	Sparkes, Sirlin, Steiner and Preining
	(and would not be possible without their contributions!) but in
	contrast to those the complete retransliteration process is
	built on the ligature functionality of \TeX\ and Metafont, thus
	effectively eliminating the need for installing any external
	preprocessor or $\Omega$mega.
\end{abstract}
\tableofcontents

\section{Introduction}

Support for Tibetan has been available for the \TeX\ and \LaTeX\
user community for quite a few years thanks to the contributions of 
Ronald Schwartz, Jeff Sparkes, Dominik Wujastyk\footnote{URL:
	\texttt{CTAN://tex-archive/language/tibetan/original/}},
Sam Sirlin\footnote{URL: 
	\texttt{CTAN://tex-archive/language/tibetan/sirlin/}}
and
Beat Steiner\footnote{URL:
	\texttt{CTAN://tex-archive/language/tibetan/steiner/}}.

\begin{sloppypar}
After studying these sources it became evident to me that these systems
were composed mainly with a \TeX\ and \LaTeX\ user in mind who is willing
to set up additional external software (the Latin--Tibetan
converter/preprocessor) and who, in addition, is willing to cope with
 decidedly ``non-\TeX-ish'' code constructs (like \verb-%%-) for marking
Tibetan language portions. I considered it useful if external preprocessors and 
unidiomatic code could be avoided at reasonable cost for the user,
i.\,e. without introducing loads of \verb-\ThisIsMySpecialTibetanLetter--like
commands. In addition, I appreciated the idea of a Tibetan support that
seamlessly fits into the \TeX\ and \LaTeXe\ world---after the typical
\verb*-\input ...- or \verb-\usepackage{...}- declaration, the user should
ideally write something like \verb-{\tib .bod skad.}- in order to obtain
\hbox{\tib .bod skad.} which implies that the retransliteration engine
is realized somewhere in the realms of \TeX\ and \LaTeX. 
\end{sloppypar}

The astute observer may inject that Norbert Preining made Tibetan support
without preprocessor available in 1997 for the $\Omega$mega system.\footnote{%
	The $\Omega$mega home is: 
	\texttt{http://www.ens.fr/omega}. $\Omega$mega is also shipped
	with recent versions of Thomas Esser's teTeX (URL:
	\texttt{http://tug.org/teTeX/}), to name just one example.
	
	Norbert Preining's adaption of Sirlin's fonts for $\Omega$mega
	can be found at:
	\texttt{http://www.logic.at/people/preining/tex/tex.html}.}
%
% <- no new paragraph here!
%
As long as
the transition to Unicode and $\Omega$mega has not been completed on a
broad basis, I consider that there is still sufficient need for a \LaTeXe-%
based Tibetan package. If its features were also usable in plain \TeX,
I decided not to frown.


\subsection{Words of Thanks}

Of course, there is D.\,E.\,Knuth who created \TeX\ in the first place,
and whose article on ``The New Versions of \TeX\ and \textsf{METAFONT}''%
\footnote{Originally published in TUGboat 10 (1989), 325--328; 11 (1990),
	12; studied in Knuth: \textit{Digital Typography}, 1999,
	CSLI Publications, Stanford, California, p. 563--570.}
ignited the initial spark for building a transliteration engine as a
pure, if not huge ligature table.
Ongoing discussions about Tibetan transliteration issues with Wolfgang Lipp
in the context of the Pentaglot Project were also very helpful.
My special thanks go to Mr.~Florian Reissinger for his patience while
reviewing the glyph lists and his comments after proof-reading the
test runs. The warmhearted patience and understanding support by
\hbox{\tib .phun tshogs bde skyid.} and \hbox{\tib .byin ba.} was
essential for finishing this project.


\subsection{Some Technical Notes}

Readers who are not interested in the internals of \TibTeX\
can safely proceed to sections~\ref{Installation} and 
\ref{UserCommands} of this text. The following notes are intended
for those who want to get an understanding of the workings behind 
the scenes.

In \TibTeX, the Tibetan retransliteration system is based on the
ligature mechanism located at the junction of Metafont and \TeX.
In order to implement the
ligature mechanism Sirlin's Metafont sources were rearranged completely.
A separate coding mechanism with symbolic character codes was introduced
so that all character interaction (numbering, ligtable programs etc.) could
be defined on a purely mnemonic basis, thus avoiding the cumbersome
inflexibility of a fixed numbering scheme. During this work a mistaken
glyph was discovered which appeared \emph{twice}: a second \emph{cwa}
({\tib cwa}) showed up in the slot of \emph{gwa} ({\tib gwa}) while
the \emph{gwa} glyph was missing altogether. A new \emph{gwa} was
then built using outline fragments of other glyphs.\footnote{%
	Alas! The mistaken \emph{cwa} which wanted to be taken for
	a \emph{gwa} even infiltrated $\Omega$mega Tibetan\dots}
Some character definitions were then taken from Norbert Preining's
work; I added a text symbol \textit{visarga} ({\tib :}) into the font.

In a further step, the vowel bounding boxes for [\emph{e i o}] were raised 
while several duplicate [\emph{u}]s with gradually sinking bounding
boxes were added. Vowels could then be combined with consonants
by simple ligature and kerning instructions, effectively eliminating
the need for any \verb-\accent--construction on the document side.
The exact alignment of carrier consonant and vowel can now be fine-tuned
in the ligature table (\verb-ctibligs.mf- in the \texttt{mfinput}
directory). While most of the conventional Tibetan words can be written
easily in \TibTeX, the system, unlike Steiner's preprocessor,
will fail with Sanskrit input as \emph{buddha} which turns out as
*{\tib buddha} instead of {\tib bu\V{d}{dh},}. In cases like this
users are requested to consult the vertical stacking command \verb|\V{}{}|
which allows explicite stacking of any desired letter combination.

Finally, a style file was created (\texttt{ctib.sty}) which provides
the user interface to the Tibetan script and related commands. The
style file calls font definition files which were created for accommodating
future extensions like new script styles and typefaces.

In early summer 2001, contact with Sam Sirlin was renewed, who had
released the version 6.0 of his Tibetan package in the meantime.
A major feature was the greatly enriched collection of glyphs
(additional ligatures, letter elements for combining new and
rare letters, diacritical marks, etc.) which was then made
available to \TibTeX.

\section{Installation\label{Installation}}

Installation of this software package is straightforward:
The installation procedure depends on the nature of the actual
\TeX\ system. The directory tree of e.\,g., teTeX is different
from the emtex tree; hence the source archive
\texttt{ctib4tex}\textit{nn}\texttt{.zip}
features the following subdirectories the contents of which has to be
placed into appropriate branches of the \TeX\ installation:
\begin{itemize}
	\item \texttt{mfinput} holds the complete Metafont sources 
		for the Tibetan fonts. The suggested path for emtex
		users is \verb"\emtex\mfinput\ctib"; for teTeX users
		\verb"$TEXMF/fonts/source/public/ctib" is a suitable
		choice.
	\item \texttt{tfm} holds all necessary font metrics files.
	The suggested path for emtex users is \verb"\emtex\tfm\ctib";
	for teTeX users \verb"$TEXMF/fonts/tfm/public/ctib"
	is a suitable choice.
	\item \texttt{texinput} holds all style files, font encoding
		definitions etc. which are read by \TeX\ and \LaTeXe.
		The suggested path for emtex users is
		\verb"\emtex\texinput\ctib"; for teTeX users
		\verb"$TEXMF/tex/latex/ctib" is a suitable choice.
	\item \texttt{doc} contains the documentation (the document
		which you are reading right now). It can be placed
		in \verb"\emtex\doc\ctib" (for emtex users) or 
		\verb"$TEXMF/doc/latex/ctib" (for teTeX users).
\end{itemize}

It may become necessary to rehash the directory database of the
\TeX\ system. When in doubt, consult your system administrator or
local \TeX\ wizard.
On teTeX systems, the command \texttt{texhash} will perform this service.


\section{User Commands\label{UserCommands}}

\LaTeXe\ users activate Tibetan support for their documents via a
\verb-\usepackage- declaration
\begin{verbatim}
	\documentclass[a4paper,11pt]{article}
	\usepackage{ctib}
\end{verbatim} in the preamble of the document, while
\TeX\ users activate Tibetan support for their documents via an
\verb*-\input - declaration
\begin{verbatim}
	\input ctib
\end{verbatim} in the beginning of the document.  The only command
necessary is \verb-\tib- which switches to Tibetan:

\exa
	This is Tibetan:\\
	{\tib \swasti. .bod skad.}
\exb
	\begin{verbatim}
	This is Tibetan:\\
	{\tib \swasti. .bod skad.}
	\end{verbatim}
\exc

The \textit{tsheg} is generated automatically after every syllable
and can be inhibited by the command \verb|\notsheg|.
\marginpar{%
	The most important diacritics:

	\textit{tsheg} inhibition via \texttt{\char92 notsheg}.

	\texttt{.} $\rightarrow$ {\tib .}\\[1.2ex]
	\texttt{,} $\rightarrow$ {\tib ,}\\[1.2ex]
	\texttt{:} $\rightarrow$ {\tib :}\\[1.2ex]
	\texttt{!} $\rightarrow$ {\tib !}\\[1.2ex]

	\raggedright
	\texttt{@} $\rightarrow$ {\tib @}\\

}
So, \verb|\tib go| creates {\tib go} whereas \verb|\tib go\notsheg|
creates {\tib go\notsheg}. The intersentence space after sentences
ending in \textit{k, g} is created by \verb|\K| which removes the
\textit{tsheg} at the same time. Additional \textit{tsheg}s are
produced by commata \verb|,| while the full stop \verb|.| generates
the \textit{shad}, the exclamation mark \verb-!- generates a
\textit{tsheg shad} and the colon \verb|:| produces a
\textit{visarga}. \verb|\swasti| can also be abbreviated as
\verb|@|. See also table~\ref{SpecialDiacritics} for a fairly complete
overview of available special symbols.

Stacks of consonants used for expressing Sanskrit words are not necessarily
contained in the basic glyph collection of \TibTeX. They can, however,
be generated easily with the \verb|\V{}{}|command (\texttt{V} like
\textit{vertical}). An abbreviation exists also for \emph{om} which is
\verb|\om|.

\exa
	{\tib \om, ma nxi pa\V{de}{ma}
	\hung \hrih:}

	Abbreviations exist:
	{\tib \dme\ \ai.}
\exb
	\begin{verbatim}
	{\tib \om, ma nxi pa\V{de}{ma}
	\hung \hrih:}

	Abbreviations exist:
	{\tib \dme\ \ai.}
	\end{verbatim}
\exc


\subsection{Transliteration Table}

Unlike with Steiner's and Preining's systems, there is only one
transliteration model available; at present the user has to accept
what the system offers. Please consult the following tables for
an overview of available symbols. Note: symbols which could be added
with this version thanks to Sam Sirlin's recent work on Tibetan
fonts were \fbox{framed} for easy identification.

\newcommand{\TF}[1]{\begin{tabular}[t]{c}{\fbox{\tib\vphantom{\hrih}%
			#1}}\\\textit{#1a}\end{tabular}}
\newcommand{\TT}[1]{\begin{tabular}[t]{c}{\vphantom{\tib \hrih}%
			\tib #1}\\\textit{#1a}\end{tabular}}
\newcommand{\TTa}[1]{\begin{tabular}[t]{c}{\vphantom{\tib \hrih}%
			\tib #1}\\\textit{#1}\end{tabular}}

\begin{table}
\begin{center}
\begin{tabular}{cccc}
	\TT{k}	&\TT{kh}	&\TT{g}		&\TT{ng}\\
	\TT{c}	&\TT{ch}	&\TT{j}		&\TT{ny}\\
	\TT{t}	&\TT{th}	&\TT{d}		&\TT{n}\\
	\TT{p}	&\TT{ph}	&\TT{b}		&\TT{m}\\
	\TT{ts}	&\TT{tsh}	&\TT{dz}	&\TT{w}\\
	\TT{zh}	&\TT{z}		&\TT{'}		&\TT{y}\\
	\TT{r}	&\TT{l}		&\TT{sh}	&\TT{s}\\
	\TT{h}	&\TTa{a}	\\
\end{tabular}
\end{center}
\caption{\TibTeX\ in Transliteration and Tibetan: Alphabet}
\end{table}

\begin{table}
\begin{center}
\TTa{a} \TTa{i} \TTa{u} \TTa{e} \TTa{o}

\end{center}
\caption{\TibTeX\ in Transliteration and Tibetan: Basic Vowels}
\end{table}

\begin{table}
\begin{center}
\TT{tx} \TT{thx} \TT{dx} \TT{nx} \TT{shx} \TT{kshx}

\end{center}
\caption{\TibTeX\ in Transliteration and Tibetan: Sanskrit letters}
\end{table}


\begin{table}
\begin{center}
Seven basic consonants with aspiration \textit{h} {\tib h} subjoined:\\
\TT{gh} \fbox{\TT{jh}} \TT{dh} \TT{bh} \TT{dzh} \TT{dxh} \TT{lh}

\bigskip Nine basic consonants with \textit{y} {\tib y} subjoined:\\
\TT{ky} \TT{khy} \TT{gy} \TT{py} \TT{phy} \TT{by} \TT{my}
\fbox{\TT{ry}} \fbox{\TT{hy}}

\bigskip Fifteen basic consonants with \textit{r} {\tib r} subjoined:\\
\TT{kr}  \TT{khr} \TT{gr} \TT{tr}  \TT{thr} \TT{dr} \TT{pr}\\
\TT{phr} \TT{br}  \TT{mr} \fbox{\TT{dzr}} \fbox{\TT{zr}}  \TT{shr} \TT{sr} \TT{hr} 

\bigskip Six basic consonants with \textit{l} {\tib l} subjoined:\\
\TT{kl} \TT{gl} \TT{bl} \TT{rl}  \TT{sl}  \TT{zl}

\bigskip Seventeen basic consonants with \textit{wazur} subjoined:\\
\TT{kw} \TT{khw} \TT{gw} \TT{cw} \fbox{\TT{chw}}  \TT{nyw}  \TT{tw} \TT{dw} \TT{tsw}\\
\TT{tshw} \TT{zhw} \TT{zw} \TT{rw} \TT{lw}  \TT{shw}  \TT{sw} \TT{hw}

\bigskip Twelve basic consonants with \textit{r} {\tib r} surmounting them:\\
\TT{rk} \TT{rg} \TT{rng} \TT{rj}  \TT{rny}  \TT{rt}\\
\TT{rd}  \TT{rn} \TT{rb} \TT{rm} \TT{rts}  \TT{rdz}

\bigskip Ten basic consonants with \textit{l} {\tib l} surmounting them:\\
\TT{lk} \TT{lg} \TT{lng} \TT{lc}  \TT{lj}
\TT{lt}  \TT{ld} \TT{lp} \TT{lb} \TT{lh} 

\bigskip Eleven basic consonants with \textit{s} {\tib s} surmounting them:\\
\TT{sk} \TT{sg} \TT{sng} \TT{sny}  \TT{st}
\TT{sd}  \TT{sn} \TT{sp} \TT{sb} \TT{sm} \TT{sts}

\end{center}
\caption{\TibTeX\ in Transliteration and Tibetan: Composites I}
\end{table}

\begin{table}
\begin{center}
\TF{bhy} \TT{nr}  \TT{dxh} \TF{drw} \TT{grw} \TT{phyw} \TT{rgw} \TT{rtsw}\\
\TT{rgy} \TT{rky} \TT{rmy} \TT{skr} \TT{sgr} \TT{smr}  \TT{spr} \TT{sbr}\\
\TT{sgy} \TT{sky} \TT{smy} \TT{spy} \TT{sby} \TF{snr}
\end{center}
\caption{\TibTeX\ in Transliteration and Tibetan: Composites II}
\end{table}



\subsection{Known Transliteration Problems\label{TransliterationProblems}}

The transliteration services of \TibTeX\ are coded in the ligature
table of the font implying that these services know a lot about adjacent
letters but know next to nothing about Tibetan syllables. The approach
follows thus always a ``first match'' and not a ``correct match'' method.
It is hence possible that consonant clusters are converted to Tibetan
glyphs in a manner which was not intended by the writer. The hyphen
\verb|-| helps solve these ambiguities:

\exa
	The syllable \emph{brtsams} is \\
	{\tib b-rtsams}, not {\tib brtsams}!
\exb
	\begin{verbatim}
	The syllable \emph{brtsams} is \\
	{\tib b-rtsams}, not {\tib brtsams}!
	\end{verbatim}
\exc


\subsection{Diacritical Marks}

Thanks to Sam Sirlin's work, this version of \TibTeX\ provides a
whole range of diacritical marks; most of them can be invoked with a 
rather lengthy command name, but for some of them there is also a
short character mnemonic. Please consult table \ref{SpecialDiacritics} on
page~\pageref{SpecialDiacritics}.

\newcommand{\TC}[2]{%
	\tt\string#1	& \tt\string#2 & {\tib#1} \\
	}

\begin{table}
\begin{center}
\begin{tabular}{llc}
	Generic Name & Alternative & Symbol \\\hline
\TC{\tibvarfive			}{}
\TC{\tibvarsix			}{}
\TC{\tibvarseven		}{}
\TC{\tibvareight		}{}
\TC{\tibvarnine			}{}
\TC{\tibempty			}{}
\TC{\tibShad			}{.}
\TC{\tibTsheg			}{,}
\TC{\tibSwasti			}{\swasti}
\TC{\tibVisarga			}{:}
\TC{\tibTshegshad		}{!}
\TC{\tibAlttshegshad		}{}
\TC{\tibNyistshegshad		}{}
\TC{\tibChemgoshad		}{}
\TC{\tibSbrulshad		}{}
\TC{\tibRgyagramshad		}{}
\TC{\tibVarchemgoshad		}{}
\TC{\tibRjessungaro		}{}
\TC{\tibSnaldan			}{}
\TC{\tibRnambcad		}{}
\TC{\tibGtertsheg		}{}
\TC{\tibRinchenspungsshad	}{}
\TC{\tibTopiniyigmgomdunma	}{(}
\TC{\tibFinalyigmgomdunma	}{)}
\TC{\tibIniyigmgomdunma		}{((}
\TC{\tibAngkhyanggyon		}{<}
\TC{\tibAngkhyanggyas		}{>}
\TC{\tibHalanta			}{}
\TC{\tibLcirtags		}{}
\TC{\tibNyizlanaada		}{}
\TC{\tibHalf			}{}
\TC{\tibGteryigmgotr		}{}
\TC{\tibPaluta			}{}
\end{tabular}
\end{center}
\caption{\TibTeX\ Special Diacritics\label{SpecialDiacritics}}
\end{table}

\section{Example}

The following text, ``The Story of \emph{Yug-pa-\`can} the  Brahman'',
is blatantly stolen from Sirlin's \texttt{/doc/} directory. Some of the
input conventions have, however, changed slightly, and so the full text
in romanized source and Tibetan target forms is given here. Please note
that the word and sentence spaces are not identical with those of Sirlin's
and Steiner's systems, hence the line and paragraph layout is not completely
identical. Modifications of input syntax are shown in the margins.

\newcommand{\Kcommand}{\texttt{\char92 K}}
\newcommand{\brahmastory}{%
	.yul zhig na bram ze dbyug pa can zhes bya zhig 'dug ste.
	rab du dbul 'phongs pa bza' ba dang,. bgo med pa zhig go.
	des khyim bdag cig las ba glang zhig b-rnyes te. nyin par
	spyad nas ba glang de khrid de khyim bdag de'i khyim du song
	ba dang,. de na khyim bdag ni zan za ste. dbyug pa can gyis ba
	glang de khyim gyi nang du btang ba dang,. ba glang sgo gzhan
	du song nas stor ro..  khyim bdag de zan de zos nas langs pa
	dang,. de na ba glang ma mthong nas des dbyug pa can la glang
	ga re zhes byas pa dang,. des smras pa. khyod kyi khyim du btang
	ngo,. .khyod kyis nga'i glang bor gyis slar byin cig ces smras
	pa dang,. des smras pa. ngas ma bor ro..  de nas de gnyis 'grogs
	te. rgyal po'i thad du 'dong ba dang,. 'u bu cag gi rigs pa dang
	mi rigs pa rtog par 'gyur ro zhes smras nas de gnyis dong ba
	dang,. mi gzhan zhig gi rta rgod ma zhig bros nas. des dbyug pa
	can la smras pa. rgod ma ma btang zhes smras pa dang,. des rdo
	zhig blangs te 'phangs pa dang rta'i rkang pa la phog nas rkang
	pa bcag go\Kcommand\marginpar{\small \texttt{\char92 K} eliminates
		the \textit{tsheg} and inserts a long space.} .des smras
	pa. khyod kyis nga'i rta bsad kyis nga'i rta byin cig\Kcommand\
	.ci'i phyir rta sbyin. des smras pa tshur shog\Kcommand\
	.rgyal po'i drung du 'dong dang,. 'u bu cag gi zhal che gcod du
	'ong ngo zhes smras nas. de dag der song ba dang,. dbyug pa can
	des 'bras par b-rtsams te. des rtsig pa zhig gi steng nas mchongs
	pa dang,. de'i drung na tha ga pa zhig thags 'thag cing 'dug pa
	de'i steng du lhung nas tha ga pa de tshe 'phos pa dang,. tha ga
	pa'i chung mas dbyug pa can de bzung nas. khyod kyis nga'i khyo
	bsad kyis nga'i khyo byin zhig ces smras pa dang,. ngas khyod kyi
	khyo ci ltar sbyin zhes smras nas. tshur shog rgyal po'i drung du
	'dong ngo,.. des 'u bu cag gi zhal ce gcad do zhes dong ba las.
	lam gyi bar na chu bo gting zab po zhig yod de. chu de'i nang
	nas tshur shing mkhan zhig te'u kha na 'khyer te 'ong ngo,. de la
	dbyug pa can gyis chu'i gting ci tsam zhes dris pa dang,. chu'i
	gting zab bo zhes smras pas ste'u chur lhung ste. ste'u ma rnyer
	pa dang,. des dbyug pa can bzung nas. khyod kyis nga'i ste'u
	chur bskyur ro.. des smras pa ngas ma bskyer ro.  .tshur shog
	rgyal po'i drung du 'dong dang,. des 'u bu cag gi zhal che gcad
	do zhes smras nas dong ngo,.  {\bit$\ldots continued \ldots$}%
}

\begin{quote}
\verb*-\swasti. - \brahmastory
\end{quote}

\renewcommand{\Kcommand}{\K}
{\tib\swasti. \brahmastory}

\section{Legal Issues}

\TibTeX\ is published under the GNU Public Licence. And very much like
I did when I picked up the work of my predecessors and changed a few
things here and there, I warmly welcome to change and improve everything,
but then: please rename the files. If not for the sake of source protection,
then at least for making sure that various \TeX\ installations around the
world do not get confused.

\section{Outlook and Desiderata}

At the time of this writing, \TibTeX\ is actually already
outdated as $\Omega$mega, the Unicode-capable \TeX\ successor, is 
available. Why, then, did I undergo this effort? I needed Tibetan now,
for ongoing Mongolian lexicographical work which is all done using
\LaTeXe\ and Mon\TeX.

Though the font provided by Schwartz, Sparkes and Sirlin is already
useful, I keep dreaming of a true Tibetan \emph{Meta}font, not just
outlines created by GNU \texttt{fontutils}. Such a genuine metafont
would greatly facilitate the creation of new and different typefaces,
at least with different weights, and the few Unicode characters not
yet covered could then be produced easily.

Though this is a work slightly more targeted at $\Omega$mega than at
\LaTeXe, I am seriously considering preparing full-fledged Native
Language Support for the standard document formats implying that all
captions and the date and number formats have to be translated into
Tibetan. With transliteration services provided internally, it should
also be possible to integrate Tibetan into the Babel system.

Anyway, whatever the mistakes and the shortcomings are that have now
crept into this Tibetan system, I can only kindly ask you to blame me,
not my predecessors any more.
\end{document}
